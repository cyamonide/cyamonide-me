% M147 2018 A1
\documentclass[12pt]{amsart}
\usepackage{amssymb}
\usepackage{tikz}

\topmargin -0.4in
\textheight 9.3in
\textwidth 6in
\oddsidemargin .25in
\pagenumbering{gobble}

\newcounter{exerci}
\renewcommand{\theexerci}{\arabic{exerci}.}
\newcounter{exercii}
\renewcommand{\theexercii}{(\arabic{exerci}) }
\newenvironment{Ex}{\begin{list}%
 {\theexerci\hfill}{\usecounter{exerci}\rightmargin=0pt\leftmargin=12pt%
 \labelwidth=12pt\labelsep=6pt\itemsep=2.5ex}}{\end{list}\medbreak}
\newenvironment{parts}{\vspace{1ex}\begin{enumerate}\renewcommand{\itemsep}{.5ex}
 \renewcommand{\labelenumi}{(\alph{enumi})}}{\end{enumerate}}
\newcommand{\hint}{{\textbf{Hint:} }}
\newcommand{\bN}{{\mathbb{N}}}
\newcommand{\bQ}{{\mathbb{Q}}}
\newcommand{\bR}{{\mathbb{R}}}
\newcommand{\bZ}{{\mathbb{Z}}}
\newcommand{\qfor}{\quad\text{for}\quad}

\begin{document}
\thispagestyle{empty}

%%%%%%   TITLE   %%%%%%
\begin{center}
\textbf{\large Math 147 (Davidson: section 2) Assignment 1}\\[1ex]
%Submit via CrowdMark by \underline{1:00 PM Friday, Sept. 14}. }\\[1ex]
Simon Liu | 20765498\\
\end{center}

\vskip 1em
\begin{Ex}

\item 

\begin{parts}
\item If all parties are innocent, whoever states that another is certainly 
guilty will be lying. Ed says that Fred did it for sure, and Ted says that at 
least one other is guilty. \textbf{Ed and Fred are lying.}
\item From Ed's statement, we deduce that Fred did it and Ted is innocent. Let 
us denote propositions that Ed, Fred, and Ted did it with $E$, $F$, and $T$ 
respectively.\\
Ed states that Fred did it and Ted is innocent, so $F \wedge \neg T$.\\
Ted's also states $\neg T$, and that at least one other is guilty, which Fred 
satisfies.\\
Lastly, since Fred states $E \Rightarrow T$ and we know $\neg T$, $\neg E$ so 
Ed is not guilty.\\
\textbf{Fred is the only guilty.}
\item We will do casework based on Ed's guilt.\\

\textit{Case 1: Ed is guilty.}\\
Since Ed lies, we have $\neg F \wedge T$.\\
Fred didn't do it, so he tells the truth. $E \Rightarrow T$.\\
Since $T$, $E \Rightarrow T$ is always true.\\
However, Ted lies, so he is not innocent and no others are guilty, which is 
an invalid statement since Ed is guilty.\\

\textit{Case 2: Ed is not guilty.}\\
Since Ed tells the truth, $F \wedge \neg T$.\\
Ted is innocent, so he states that he himself is innocent - a valid statement. 
We can also deduce from Ted that at least one other person is guilty, which is
true if Fred is guilty.\\
Fred states that Ed's guilt does not imply Ted's. Ed is not guilty, so Fred's 
statement has no effect.\\
So, \textbf{Fred is the only guilty person.}

\end{parts}

\item Let $\theta = \alpha$ where $\sin{\alpha} = 5/13$ and $\cos{\alpha} = 12/13$. 
We seek to show that this is when equality holds.
\begin{align*}
	|5 \sin{\alpha} + 12 \cos{\alpha}| &\le 13\\
	|5(5/13) + 12(12/13)| &\le 13\\
	\left|\frac{169}{13}\right| &\le 13\\
	13 &= 13
\end{align*}

\pagebreak

Now, let us express $\theta$ as angle $\alpha + \beta$ for some angle $\beta$. 
Now, we have:
\begin{align*}
	|5\sin{(\alpha + \beta)} + 12\cos{(\alpha + \beta)}| &\le 13\\
	|5(\sin{\alpha}\cos{\beta} + \sin{\beta}\cos{\alpha}) + 12(\cos{\alpha}\cos{\beta} - \sin{\alpha}\sin{\beta})| &\le 13\\
	\left|5\left(\frac{5}{13}\cos{\beta} + \frac{12}{13}\sin{\beta}\right) + 12\left(\frac{12}{13}\cos{\beta} - \frac{5}{13}\sin{\beta}\right)\right| &\le 13\\
	\left|\frac{25}{13}\cos{\beta} + \frac{60}{13}\sin{\beta} + \frac{144}{13}\cos{\beta} - \frac{60}{13}\sin{\beta}\right| &\le 13\\
	|13\cos{\beta}| &\le 13\\
	|\cos{\beta}| &\le 1\\
\end{align*}
which is true.

\item Let $A = \log{a}$, $B = \log{b}$, and $C = \log{c}$.
\begin{align*}
	(\log_a{bc})(\log_b{ac})(\log_c{ab}) &= 
	\left(\frac{\log{b}+\log{c}}{\log{a}}\right)
	\left(\frac{\log{a}+\log{c}}{\log{b}}\right)
	\left(\frac{\log{a}+\log{b}}{\log{c}}\right)\\
	&= \left(\frac{B+C}{A}\right)\left(\frac{A+C}{B}\right)\left(\frac{A+B}{C}\right)\\
	&= \frac{2ABC + A^2B + AB^2 + B^2C + BC^2 + A^2C + AC^2}{ABC}\\
	&= 2 + \frac{A^2B + AB^2}{ABC} + \frac{B^2C + BC^2}{ABC} + \frac{A^2C + AC^2}{ABC}\\
	&= 2 + \frac{A+B}{C} + \frac{B+C}{A} + \frac{A+C}{B}\\
	&= 2 + \log_a{bc} + \log_b{ac} + \log_c{ab}
\end{align*}

\item \begin{itemize}

\item[(a)\;] \begin{align*}
	\frac{2ah}{4a^2+h} < \sqrt{a^2+h} - a < \frac{h}{2a}\\
\end{align*}
\begin{align*}
	\sqrt{a^2+h}-a &= \frac{a^2+h-a^2}{\sqrt{a^2+h}+a}\\
	&= \frac{h}{\sqrt{a^2+h}+a}\\
\end{align*}

\pagebreak

Solving the upper bound,

\begin{align*}
	\frac{h}{\sqrt{a^2+h}+a} &< \frac{h}{2a}\\
	2a &< \sqrt{a^2+h}+a\\
	a^2 &< a^2+h\\
	0 &< h.
\end{align*}

For the lower bound,

\begin{align*}
	\frac{2ah}{4a^2+h} &< \frac{h}{\sqrt{a^2+h}+a}\\
	2a\sqrt{a^2+h}+2a^2 &< 4a^2+h\\
	a^2+h &< \left(\frac{2a^2+h}{2a}\right)^2\\
	a^2+h &< \frac{4a^4+h^2+4a^2h}{4a^2}\\
	4a^4+4a^2h &< 4a^4 + h^2 + 4a^2h\\
	0 &< h^2.
\end{align*}

\item[(b)\;] \begin{align*}
	b-\sqrt{b^2-k} &= -(\sqrt{b^2-k}-b).
\end{align*}

From 4a,

\begin{align*}
	\frac{2ah}{4a^2+h} &< \sqrt{a^2+h}-a < \frac{h}{2a}\\
	\frac{-2ah}{4a^2+h} &> a - \sqrt{a^2+h} > \frac{-h}{2a}\\
\end{align*}

Letting $a=b$ and $\sqrt{a^2+h} = \sqrt{b^2-k} \Rightarrow h=(-k)$,

\begin{align*}
	\frac{-2b(-k)}{4b^2+(-k)} &> b-\sqrt{b^2-k} > \frac{-(-k)}{2b}\\
	\frac{2bk}{4b^2-k} &> b-\sqrt{b^2-k} > \frac{k}{2b}.
\end{align*}

\item[(c)\;] $\Delta_a = \sqrt{a^2+8}-a \rightarrow \text{(4 a) where } h=8$.
\begin{align*}
	\frac{2a(8)}{4a^2+8} &< \Delta_a < \frac{8}{2a}\\
	\frac{4a}{a^2+2} &< \Delta_a < \frac{4}{a}\\
	\frac{a^2+2}{4a} &> \frac{1}{\Delta_a} > \frac{a}{4}\\
	\frac{a}{4} + \frac{1}{2a} &> \frac{1}{\Delta_a} > \frac{a}{4}.
\end{align*}

$\Delta_b = b - \sqrt{b^2-1} \rightarrow \text{(4 b) where } k=1$.
\begin{align*}
	\frac{2b(1)}{4b^2-1} &> \Delta_b > \frac{1}{2b}\\
	\frac{4b^2-1}{2b} &< \frac{1}{\Delta_b} < 2b\\
	2b - \frac{1}{2b} &< \frac{1}{\Delta_b} < 2b.
\end{align*}

Subtracting $\frac{1}{\Delta_b}$ from $\frac{1}{\Delta_a}$,
\begin{align*}
	\frac{a}{4}-2b+\frac{1}{2b} < \frac{1}{\Delta_a} - \frac{1}{\Delta_b} &< \frac{a}{4} + \frac{1}{2a} - 2b\\
	\frac{1}{2b} < \left(\frac{1}{\Delta_a} - \frac{1}{\Delta_b}\right) - \left(\frac{a}{4} - 2b\right) &< \frac{1}{2a}\\
\end{align*}

Since $\frac{a}{4} =246913580.25$ and $2b = 246913578$, $\frac{a}{4} > 2b$, so $\left(\frac{a}{4}-2b\right) > 0$. Call this $m = \frac{a}{4} - 2b$.\\
Our lower bound $\frac{1}{2b}$ is positive, so the middle expression must be positive. Note that we are subtracting a positive $m$ from $\frac{1}{\Delta_a} - \frac{1}{\Delta_b}$, so it follows that $\frac{1}{\Delta_a} - \frac{1}{\Delta_b}$ is positive, and
\begin{align*}
	\frac{1}{\Delta_a} &> \frac{1}{\Delta_b}\\
	\Delta_a &< \Delta_b,\\
\end{align*}
so \textbf{$\Delta_b$ is larger.}\\

I have not yet found a way to show that $|\Delta_a = \Delta_b| < 5\cdot 10^{-11}$ without substituting $a$ and $b$ and taking the difference of their individual bounds to get $|\Delta_a = \Delta_b| < 3.7\cdot 10^{-17}$.

\end{itemize}

\pagebreak

\item \begin{parts}
\item Let us split the real line into four disjoint intervals $x < -1$, $-1 < x < 0.5$, $0.5 < x < 1$, and $1 < x$.\\

When $x < -1$, 
\begin{align*}
	f(x) &= |x-|x-1|| - |x-|x+1||\\
	&= |x + x - 1| - |x + x + 1|\\
	&= |2x - 1| - |2x + 1|\\
	&= 1 - 2x + 2x + 1\\
	&= 2
\end{align*}

When $-1 < x < 0.5$, 
\begin{align*}
	f(x) &= |x-|x-1|| - |x-|x+1||\\
	&= |x + x - 1| - |x - x - 1|\\
	&= |2x - 1| - |-1|\\
	&= 1 - 2x - 1\\
	&= -2x
\end{align*}

When $0.5 < x < 1$, 
\begin{align*}
	f(x) &= |x-|x-1|| - |x-|x+1||\\
	&= |x + x - 1| - |x - x - 1|\\
	&= |2x - 1| - |-1|\\
	&= 2x - 1 - 1\\
	&= 2x - 2
\end{align*}

When $1 < x$, 
\begin{align*}
	f(x) &= |x-|x-1|| - |x-|x+1||\\
	&= |x - x + 1| - |x - x - 1|\\
	&= |1| - |-1|\\
	&= 1 - 1\\
	&= 0
\end{align*}

\pagebreak

\item $f(x)$ is not differentiable at $x = -1.0$, $x = 0.5$, and at $x = 1.0$ because $\forall a \in \{-1.0, 0.5, 1.0\}$, $\lim_{x \to a^+}f'(x) \neq \lim_{x \to a^-}f'(x)$.\\

\begin{tikzpicture}[scale=2]
	\draw[->] (-3,0) -- (3,0) node[right] {$x$};
	\draw[->] (0,-3) -- (0,3) node[above] {$y$};
	\draw[domain=-3:-1,smooth,variable=\x, red] plot ({\x},{2});
	\draw[domain=-1:0.5,smooth,variable=\x, red] plot ({\x},{-2*\x});
	\draw[domain=0.5:1,smooth,variable=\x, red] plot ({\x},{2*\x-2});
	\draw[domain=1:3,smooth,variable=\x, red] plot ({\x},{0});
	\draw[fill] (-2,0) circle [radius=0.025];
	\draw[fill] (-1,0) circle [radius=0.025];
	\draw[fill] (0,0) circle [radius=0.025];
	\draw[fill] (1,0) circle [radius=0.025];
	\draw[fill] (2,0) circle [radius=0.025];
	\draw[fill] (0, -2) circle [radius=0.025];
	\draw[fill] (0, -1) circle [radius=0.025];
	\draw[fill] (0, 0) circle [radius=0.025];
	\draw[fill] (0, 1) circle [radius=0.025];
	\draw[fill] (0, 2) circle [radius=0.025];
	\node [below] at (-1,0) {-1};
	\node [above] at (1,0) {1};
	\node [right] at (0,1) {1};
	\node [right] at (0,2) {2};
	\node [left] at (0,-1) {-1};
\end{tikzpicture}

\end{parts}

\item We seek to prove that if $d \in \bN$ is not a perfect square, then $\sqrt{d}$ is irrational by contradiction.\\

$\exists m \in \bZ^+$ s.t. $m^2 < d < (m+1)^2$.\\

Assume that $\sqrt{d} \in \bQ$.\\

If $\sqrt{d} \in \bQ$, let $A = \{n \in \bN : n\sqrt{d} \in \bN\}$.\\

Also, let $a$ be the smallest member of $A$, and $b = \sqrt{d} - am$.\\

First, observe that
\begin{align*}
	m^2 < d < (m+1)^2 &\Rightarrow m < \sqrt{d} < m+1\\
	&\Rightarrow 0 < \sqrt{d}-m < 1.
\end{align*}

So,
\begin{align*}
	b &= a\sqrt{d}-am\\
	&=a(\sqrt{d}-m)\\
	&\Rightarrow b<a.
\end{align*}

Also, since $a\sqrt{d} \in \bN$ (by definition of set $A$), $a \in \bN$, and $m \in \bZ^+$, we know that $b=a\sqrt{d}+am\in\bN$.\\

We now seek to show that $b \in \bN$ belongs in $A$.

\begin{align*}
	b\sqrt{d} &= (a\sqrt{d}+am)\sqrt{d}\\
	&= ad + (a\sqrt{d})m
\end{align*}

Since all $ad$, $a\sqrt{d}$, and $m$ are elements of $\bN$, $b\sqrt{d} \in \bN$. So, $b$ must be an element of $A$.\\

Originally, we assumed that $a$ is the least element in $A$. However, $b<a$ and $b \in A$. This is impossible.\\

$\therefore \sqrt{d} \notin \bQ$ by contradiction. $\qed$

\end{Ex}
\end{document}
