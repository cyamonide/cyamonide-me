\documentclass[12pt]{article}
\usepackage{amsmath}
\usepackage{amssymb}
\usepackage{amsthm}

\topmargin -0.4in
\textheight 9.3in
\textwidth 6in
\oddsidemargin .25in
\setlength\parindent{0pt}

\newcommand{\bN}{{\mathbb{N}}}
\newcommand{\bQ}{{\mathbb{Q}}}
\newcommand{\bR}{{\mathbb{R}}}
\newcommand{\bZ}{{\mathbb{Z}}}
\newcommand{\Mod}[1]{\ (\mathrm{mod}\ #1)}

\newtheorem{theorem}{Theorem}[section]
\newtheorem{corollary}{Corollary}[theorem]
\newtheorem{lemma}[theorem]{Lemma}
\newtheorem{definition}{Definition}[section]
\newtheorem{conj}[theorem]{Conjecture}

\begin{document}
\thispagestyle{empty}

%%%%%%   TITLE   %%%%%%
\begin{center}
\textbf{\large Math 145 (Jao: section 1) A2 Numerical}\\[1ex]
%Submit via CrowdMark by \underline{1:00 PM Friday, Sept. 14}. }\\[1ex]
Simon Liu | 20765498\\
\end{center}

\vskip 1em

\setcounter{section}{6}

\section{Numerical Problem 1}

	\subsection{Finding $-1$ in $\bZ_n$} \

		Axiom $A4$ states that $a + (-a) = 0$. \\

		$\bZ_5$:

		\begin{align*}
			1 + (-1) &= 0\\
			1 + 4 &= 0\\
			\therefore -1 &= 4.
		\end{align*}

		In integer systems $\bZ_6$, $\bZ_7$, and $\bZ_10$, 

		\begin{table}[h!]
			\centering
			\begin{tabular}{c|c|c|c|c}
				& $\bZ_5$ & $\bZ_6$ & $\bZ_7$ & $\bZ_{10}$ \\
				\hline
				\hline
				$-1$ & $4$ & $5$ & $6$ & $9$\\
			\end{tabular}
		\end{table}

		\begin{theorem}
			$-1 = n - 1$ in $\bZ_n$.
		\end{theorem}

		See proof for Theorem 7.2 with $a = 1$.

		\begin{theorem}
			$-a = n - a$ in $\bZ_n$.
		\end{theorem}

		\begin{proof}
			For all $a$ and $k$ in $\bZ$,
			\begin{align*}
				a &\equiv kn + a \Mod{n}\\
				-a &\equiv -kn - a \Mod{n}\\
			\end{align*}

			Letting $k = -1$,
			\begin{align*}
				-a &\equiv n - a \Mod{n}\\
			\end{align*}

			We know that
			\begin{align*}
				\forall m \in \bZ, m \Mod{n} \in \bZ_n
			\end{align*}

			So,
			\begin{align*}
				-a &\equiv n - a \Mod{n} \in \bZ_n\\
			\end{align*}
		\end{proof}

	\subsection{Finding $\frac{1}{2}$ in $\bZ_n$} \

		\begin{definition}
			$\frac{1}{a}$ is the element of $\bZ_n$ satisfying $a \cdot \frac{1}{a} = 1, \forall a \in \bZ_n$ if it exists.
		\end{definition}

		For $\frac{1}{2}$, $a=2$ in Definition 7.1.\\

		$\bZ_5$:

		\begin{align*}
			2 \cdot \frac{1}{2} &\equiv 1 \equiv 6 \Mod{5}\\
			\frac{1}{2} &= 3 \in \bZ_5
		\end{align*}

		$\therefore$ $\frac{1}{2}$ exists in $\bZ_5$.\\
		
		$\bZ_6$:

		\begin{align*}
			2 \cdot \frac{1}{2} &\equiv 7 \Mod{6}\\
		\end{align*}

		However, there is no integer $b$ where $2 \cdot b = 7$, so we try larger products that are congruent to $1 \Mod{6}$, only to observe that the product (R.S. of equation) is always odd. There is no such $a \in \bZ_6$ that when multiplied by $2$ results in an odd integer.\\

		$\therefore$ $\frac{1}{2}$ does not exist in $\bZ_5$.\\

		Working more examples,

		\begin{table}[h!]
			\centering
			\begin{tabular}{c|c|c|c|c|c|c}
				& $\bZ_5$ & $\bZ_6$ & $\bZ_7$ & $\bZ_{8}$ & $\bZ_{9}$ & $\bZ_{10}$ \\
				\hline
				\hline
				$\frac{1}{2}$ & Exists & DNE & Exists & DNE & Exists & DNE
			\end{tabular}
		\end{table}

		\begin{theorem}
			$\frac{1}{2}$ does not exist in $\bZ_n$ when $\bZ_n$ is even.
		\end{theorem}

		See proof for Theorem 7.4 with $k = 2$.

	\pagebreak

	\subsection{Finding $\frac{1}{3}$ in $\bZ_n$}

		$\bZ_5$:

		\begin{align*}
			3 \cdot \frac{1}{3} &\equiv 1 \equiv 6 \Mod{5}\\
			\frac{1}{3} &= 2 \in \bZ_5
		\end{align*}

		$\therefore$ $\frac{1}{3}$ exists in $\bZ_5$.\\

		$\bZ_6$:

		\begin{align*}
			3 \cdot \frac{1}{3} &\equiv 1 \equiv 7 \equiv 13 \Mod{6}\\
		\end{align*}

		Similar to how $\frac{1}{2} \notin \bZ_6$ in Section 7.2, $\frac{1}{3}$ does not seem to exist in $\bZ_6$ either.\\

		Working more examples,

		\begin{table}[h!]
			\centering
			\begin{tabular}{c|c|c|c|c|c|c}
				& $\bZ_5$ & $\bZ_6$ & $\bZ_7$ & $\bZ_{8}$ & $\bZ_{9}$ & $\bZ_{10}$ \\
				\hline
				\hline
				$\frac{1}{2}$ & Exists & DNE & Exists & Exists & DNE & Exists
			\end{tabular}
		\end{table}

		\begin{table}[h!]
			\centering
			\begin{tabular}{c|c|c|c|c|c|c}
				& $\bZ_6$ & $\bZ_9$ & $\bZ_{12}$ & $\bZ_{15}$ & $\bZ_{18}$ & $\bZ_{21}$ \\
				\hline
				\hline
				$\frac{1}{2}$ & DNE & DNE & DNE & DNE & DNE & DNE
			\end{tabular}
		\end{table}

		Generalizing Theorem 7.3,

		\begin{theorem}
			$\frac{1}{k}$ does not exist in $\bZ_n$ when $k | n$
		\end{theorem}

		\begin{proof}
			By Definition 7.1, letting $a = \frac{1}{k}$,
			\begin{align*}
				k \cdot a \equiv 1 \Mod{n}.
			\end{align*}

			Suppose that $k|n$, ie. $n = k \cdot m, m \in \bZ^+$. Then,

			\begin{align*}
				k \cdot a &\equiv 1 \Mod{mk}\\
				k \cdot a &\equiv 1 \Mod{k}\\
				k \Mod{k} \cdot a \Mod{k} &\equiv 1 \Mod{k}\\
				0 \cdot a \Mod{k} &\equiv 1 \Mod{k}
			\end{align*}

			This is not possible for the non-trivial values of $k>1$, therefore $a$ does not exist.
		\end{proof}

\pagebreak

	\subsection{Finding $\frac{1}{k}$ in $\bZ_n$}

		After working through more examples on varying values of $k$ and $n$, it seemed that $k$ and $n$ had to be coprime for $\frac{1}{k}$ to exist.\\

		Here are some notable examples.

		\begin{table}[h!]
			\centering
			\begin{tabular}{c|c|c|c}
				& $\bZ_7$ & $\bZ_{10}$ & $\bZ_{12}$  \\
				\hline
				$1/1$ & 1     & 1      & 1      \\
				$1/2$ & 4     & DNE    & DNE     \\
				$1/3$  & 5     & 7      & DNE    \\
				$1/4$  & 2     & DNE    & DNE    \\
				$1/5$  & 3     & DNE    & 5      \\
				$1/6$  & 6     & DNE    & DNE    \\
				$1/7$  & DNE   & 3      & 7      \\
				$1/8$  &       & DNE    & DNE    \\
				$1/9$  &       & 9      & DNE    \\
				$1/10$ &       & DNE    & DNE    \\
				$1/11$ &       &        & 11     \\
				$1/12$ &       &        & DNE     
			\end{tabular}
		\end{table}

		\begin{conj}
			$\frac{1}{k}$ does not exist in $\bZ_n$ when $gcd(k, n) > 1$.
		\end{conj}

	\subsection{Finding $\sqrt{-1}$ in $\bZ_n$}

		\begin{definition}
			$\sqrt{a}$ is an element of $\bZ_n$ satisfying $(\sqrt{a})^2 = a$ if it exists.
		\end{definition}

		When looking for $\sqrt{-1}$, we need to look for a value $a$ when squared results in $n-1$ by Theorem 7.1. \\

		I listed examples shown in the spreadsheet 'squares.ods' (uploaded to Learn) on Sheet 1. The second row is values of $n$ representing integer systems $\bZ_n$, and the first column is values of a to square. Entries in the table compute $a^2 \Mod{n}$.\\

		If $\sqrt{-1}$ exists in any given $\bZ_n$, then in its respective column there will exist a cell equal to $n-1$. These values are highlighted in yellow.\\

		Hoping to find a visual pattern for the existence of $\sqrt{-1}$, I did not reach any conclusions. I did however notice that for any given $\bZ_n$, $a^2 \Mod{n}$ was symmetrical across $a$.

\pagebreak

		\begin{theorem}
			$\forall a \in \bZ_n, a^2 \equiv (n-a)^2 \Mod{n}$
		\end{theorem}

		\begin{proof}
			\begin{align*}
				a^2 &\equiv (n-a)^2 \Mod{n}\\
				a^2 &\equiv n^2 + a^2 - 2na \Mod{n}\\
				a^2 \Mod{n} &\equiv n^2 \Mod{n} + a^2 \Mod{n} - 2na \Mod{n}\\
				a^2 \Mod{n} &\equiv 0 + a^2 \Mod{n} - 0\\
				a^2 &= a^2 \Mod{n}
			\end{align*}
		\end{proof}

		I did, however, have a list of $n \le 100$ where $\sqrt{-1}$ exists in $\bZ_n$. Call this sequence $S_i$.

		\begin{align*}
			S_i : \{1, 2, 5, 10, 13, 17, 25, 26, 29, 34, 37, 41, 50, 53, 58, 61, 65, 73, 74, 82, 85, 89, 97\}
		\end{align*}

		In majority of the above number systems, there existed two values of $\sqrt{-1}$. Interestingly, $n=65$ and $n=85$ had four. There were some patterns I could recognize, however none of them described existence of $\sqrt{-1}$ completely.

		\begin{theorem}
			$\sqrt{-1}$ exists in $\bZ_n$ when $n = k^2 +1, k \in \bZ$.
		\end{theorem}

		\begin{proof}
			\begin{align*}
				\sqrt{-1} &= k \in \bZ_{k^2+1}\\
				\sqrt{-1} &\equiv k \Mod{k^2+1}\\
				(\sqrt{-1})^2 &\equiv k^2 \Mod{k^2+1}\\
				-1 &= k^2 \Mod{k^2+1}\\
			\end{align*}

			This is true by Theorem 7.1:

			\begin{align*}
				-1 &\equiv n-1 \Mod{n}\\
				1 \equiv k^2 &\equiv (k^2 + 1)-1 \Mod{n}\\
				k^2 &\equiv k^2 \Mod{n}\\
			\end{align*}
		\end{proof}

		With no additional insight, I entered the first few numbers of $S_i$ into The On-Line Encyclopedia Of Integer Sequences (OEIS), finding sequence A008784 to be what I was looking for.\\

\pagebreak

		An interesting property of this sequence (other than that it represents $n$ s.t. $\sqrt{-1}$ exists $\Mod{n}$) is that every element could be represented as a sum of squares. However, this did not mean that I could simply state that $\forall a b \in \bZ, a^2 + b^2 \in S_i$.\\

		This, in turn, meant that it was not as straightforward as I initially thought it would be to construct elements of $S_i$. I needed to recognize another pattern to do so.\\

		By creating a table in Sheet 2 of 'squares.odt', I computed the sums of squares for $a, b \le 10$. It seemed that $a$ and $b$ had to be coprime for the sum of their squares to belong in $S_i$.

		\begin{conj}
			$\forall a, b \in \bZ$, if $gcd(a, b)=1$, then $\sqrt{-1}$ exists in $\bZ_n$ where $n = a^2 + b^2$.
		\end{conj}

		However, I could not explain the anomaly of there existing four or more values of $\sqrt{-1}$ in $\bZ_{65}$ and $\bZ_{85}$. Theorem 7.6 seemed to explain there existing two values of $\sqrt{-1}$ to some extent, but what caused four or more values?\\

		I went to Prof. Jao's office hours, and he pointed out a particular property of modular arithmetic:

		\begin{definition}
			$a \equiv b \Mod{pq} \rightleftharpoons a \equiv b \Mod{p} \wedge a \equiv b \Mod{q}$.
		\end{definition}

		Lending itself to:

		\begin{theorem}
			If $\sqrt{-1}$ exists in $\bZ_n$ and $\bZ_m$, then it exists in $\bZ_{nm}$.
		\end{theorem}

		Now, with Conjecture 7.8 and Theorem 7.9, we can see that $65 = 4^2 + 7^2$, and is also the product of $5$ and $13$, both of which are values of $n$ where $\sqrt{-1}$ exists in $\bZ_n$. Similarly, $85 = 5^2 + 8^2$, and $85 = 5 * 17$.

\pagebreak

\section{Numerical Problem 2}

	\subsection{Computation of $\left| \alpha - \frac{41}{24} \right|$}

		\begin{table}[h!]
			\centering
			\begin{tabular}{c|c|c}
				$\alpha$ & $\alpha - (41/24)$ & $\left| \alpha - (41/24)\right|$ \\
				\hline
				1/1      & -17/24             & 17/24                          \\
				2/1      & 7/24               & 7/24                           \\
				5/3      & -1/24              & 1/24                           \\
				12/7     & 1/168              & 1/168                          \\
				41/24    & 0                  & 0                             
			\end{tabular}
		\end{table}

		Notice that the sequence $\{\alpha - \frac{41}{24}\}$ seems to converge to 0, after having terms alternate between being positive and negative. $\{\left|\alpha - \frac{41}{24}\right|\}$ also converges to 0.

	\subsection{An attempt at defining a pattern}

		Let us define $S_A : \{0, 1, 3, 7, 17, 24, 41\}$, and $A=7$ to be the size of $S_A$.\\
		$a_n \in S_A$ where $a_1 = 0$, $a_2 = 1$, etc.\\

		Similarly, we define $S_B : \{1, 1, 2, 2, 3\}$ with respective $B=5$ and $b_n \in S_B$.

		To express the general term $a_n$ in terms of these two sets, we have:

		\begin{align*}
			a_n = a_{n-1} \cdot b_{B-n+3} + a_{n-2}
		\end{align*}

		So, to construct members of $S_A$, the first two elements $a_1$ and $a_2$ must be given, as well as a full set $S_B$ where $B = A-2$.

	\subsection{Observations from the table}

		\begin{table}[h!]
			\centering
			\begin{tabular}{c|c|c|c|c|c|c||c|}
				&   & 1 & 1 & 2 & 2  & 3 & $b$ \\
			\hline
			0 & 1 & 1 & 2 & 5 & 12 & 41 & $c$\\
			\hline
			1 & 0 & 1 & 1 & 3 & 7  & 24 & $a$\\
			\hline
			\end{tabular}
		\end{table}

		From the patterns suggested in the assignment,

		\begin{align*}
			c = b \cdot 41 + 12
		\end{align*}

		More generally, a term in the row containing $c$ is the product of the term immediately above it with the term to the left, plus the term two to the left.\\

		Furthermore, we can say that
		
		\begin{align*}
			a \cdot 41 - c \cdot 24 = 1
		\end{align*}

\end{document}